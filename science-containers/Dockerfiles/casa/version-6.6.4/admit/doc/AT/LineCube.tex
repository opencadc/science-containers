% in RST now, don't edit here.

\subsection{LineCube\_AT}

\subsubsection{Description}

A LineCube is a small spectral cube cut from the full spectral window cube,
typically centered in frequency/velocity on a given spectral line.
LineCube\_AT creates one (or more) of such cubes. It is done
after line identification (using a LineList\_BDP), so the appropriate channel ranges are
known.  Optionally this AT can be used without the line identification, and
just channel ranges are given (and thus the line will be designated as something
like ``U-115.27''.
The line cubes are normally re-gridded onto a
common velocity (km/s) scale, for later comparison,
and thus the CubeStats\_AT will need to be re-run
on these cubes. Also, normally the continuum will have been subtracted
these should only be the line emission, possibly with contamination from
other nearby lines. This is a big issue if there is a forest of lines,
and proper separation of such lines is a topic for the future and may be 
documented here as well with a method.

One of the goals of LineCube\_AT is to creates identically sized cubes of
different molecular transitions for easy comparison and cross referenced analysis.

\subsubsection{Use Case}
A LineCube would be used to isolate a single molecular component in frequency dimensions. The produced subimage would be one of the natural inputs for the Moment\_AT task.

\subsubsection{Input BDPs}

\begin{description}
\item[SpwCube\_BDP] a (continuum subtracted) spectral window cube

\item[LineList\_BDP] a LineList, as created with LineID\_AT. This LineList does not need
to have identified lines. In the simple version of LineID will just designate
channel ranges as ``U'' lines.

\item[CubeStats\_BDP]  (optional) CubeStats associated with the input spectral cube.
this is useful if the RMS noise would depend on the channel, which can happen for wider cubes,
especially near the spectral window edges. Otherwise the keyword {\bf cutoff=} will suffice.
\end{description}

\subsubsection{Input Keywords:}

\begin{description}
\item[regrid]* The velocity to regrid the output cube channels to, units: can be specified as other CASA values are (3.0kms), but defaults to km/s [None] (i.e. no regridding)
\item[cutoff]* The cutoff value in sigma to use when generating the subcube(s), all values below this will be masked [5.0]
\end{description}
* denotes an optional keyword and [] gives the default value

\subsubsection{Output BDPs:}

\begin{description}
\item[LineCube\_BDP]
The output from this AT will be a line cube for each input spectral line.
Each spectral cube will be centered on the spectral line and regridded to a common velocity scale, if requested. 
\end{description}

% this is a new section. Either we separate them, or we fold them into each output BDP.
% but this is where we describe what kind of visual outputs for the GUI it will create
\subsubsection{Output Graphics:}

none.

In some sense LineCube\_AT is like Ingest\_AT, it creates one (or more) SpwCube\_BDP's, 
and they typically do not have a visual cue for the GUI (currently).


\subsubsection{Procedure}
For each line found in the input LineList\_BDP this AT will grab a subcube based on the input parameters.

This AT will use the following CASA tasks:
\begin{description}
\item[imsubimage] to extract the final subcube from the main cube, including any masking
\item[imregrid] if regridding in velocity is needed
\end{description}

%A discussion on the need for regridding to velocity space ({\bf regrid=} keyword)
%is in place here: for comparing nearby lines in the 
%same spectral window probably not. Accross a 2 GHz window, marginal at 100GHz.
%Between spectral windows, probably yes.  It should also be noted that, given that {\bf clean}
%is a (not well understood) non-linear process, the best way to create cubes that can be
%compared is an iterative process. Find out where the lines are from the first iteration,
%go back to the visibility data, and clean those with the requested choice of 
%channels in velocity space.  This would then create LineCube's that can be compared.
%This is beyond the scope of the current ADMIT.

\clearpage
