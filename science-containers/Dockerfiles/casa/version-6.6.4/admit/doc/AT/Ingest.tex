% converted to RST now, don't edit this anymore.

\subsection{Ingest\_AT}

\subsubsection{Description}

Ingest\_AT converts a FITS file into a CASA Image.
It is possible to instead keep this a
symlink to the FITS file, as some CASA tasks are able to deal with
FITS files directly, potentially cutting down on the I/O overhead.
During the conversion to a CASA Image a few extra processing
steps are possible, e.g.
primary beam correction and taking a subcube instead of the
full cube. Some of these steps may add I/O overhead.


\subsubsection{Use Case}


Normally this is the first step that any ADMIT needs to do
to set up a flow for a project, and this is done by ingesting
a FITS file into a CASA Image. The SpwCube\_BDP contains
merely the filename of this CASA Image.



\subsubsection{Input BDPs}

none. This is an exceptional case (for bootstrapping) where a (FITS) filename
is turned into a BDP.

\subsubsection{Input Keywords:}

\begin{description}

\item[file] The filename of the  FITS file. No default, this is the only required
keyword. Normally 
with extension ``.fits'', we will refer to this as ``basename.fits''. 
% We leave the issue of absolute/relative/symlinked names up to the caller.
We will allow this to be a casa image as well.

\item[pbcorr] Apply a primary beam correction to the input FITS file? If true,
in CASA convention, a ``basename.flux.fits'' needs to be present. Normally,
single pointings from ALMA are not primary beam corrected, but mosaiced pointings
are and thus would need such a correction to make a noise flat image.
{\it We should probably allow a way to set another filename }

\item[region] Select a subcube for conversion, using the CASA region notation. By
default the whole cube is converted. See also {\bf edge=} below.

\item[box] Instead of the more complex ``region'' selection described above, a
much simpler ``blc,trc'' type description is more effective.

{\it Discussion about region= vs. box=, and is edge= still needed?}

\item[edge] Edge channel rejections. Two numbers can be given, removing the first
and last selected number of channel. Default: 0,0  {\it This keyword could
interfere with {\bf region=}, but is likely the most common one used}

\item[mask] If True a mask also needs to be created. 
% need some discussion on the background of this curious option
% importfits also has a somewhat related zeroblanks= keyword

\item[symlink]  True/False: if True, a symlink is kept to the fitsfile,
  instead of conversion to a CASA Image.
  This may be used in those rare cases where your whole flow can work
  with the fits file, without need to convert to a CASA Image
  Setting to True, also disables all other processing
  (mask/region/pbcorr). This keyword should be used with caution,
  because existing flows will then break if another task is added that
  cannot use a FITS file.
  [False]
% theoretically File_AT can also be used to create a File_BDP with 
% a symlinked FITS file, but will subsequent AT's recognize this
% underloaded BDP ?

\item[type]  If you ingest an ``alien'' object of which the origin
  is not clear, there must be a way to set the type of image
  (line, continuum, moment, alpha, ...)

\end{description}


\subsubsection{Output BDPs:}

\begin{description}

\item[SpwCube\_BDP] the spectral window cube.
The default filename for this CASA Image is ``cim''.
There are no associated graphical cues for this BDP. Subsequent steps,
such as CubeSpectrum\_AT and CubeStats\_AT. In the case that the input
file was already a CASA image, and no subselection or PB correction
was done, this can be a symlink to the input filename.

\end{description}

\subsubsection{Procedure}

Ignoring the symlink option (which has limited use in a full flow),
CASA's importfits will first need to create a full copy of the FITS
file into a CASA Image, since there are no options to process a
subregion. In the case of multiple CASA programs processing (e.g. when
region selection plus primary beam correction), it may be useful to
check the performance difference between using tools and tasks.

Although using {\bf edge=} can be a useful operation to cut down the
filesize a little bit, the catch-22 here is that - unless the user
knows - the bad channels are not known yet until CubeStats\_AT has
been run, which is the next step after Ingest\_AT. Potentially one can
re-run Ingest\_AT.

\subsubsection{CASA tasks used}

\begin{description}

\item[importfits] if just a conversion is done. No region selection
  can be done here, only some masking operation to replace masked
  values with 0s.

\item[imsubimage] only if region processing is done.  Note box= and
  region= and chans= have common methods, our Ingest\_AT should keep
  that simpler.  For example, chans="11~89" for a 100 channel cube
  would be same as edge=11,10

\item[immath] if a primary beam correction is needed, to get noise
  flat images.

\end{description}

Alternatively, AT's can also use the CASA tools. Especially this can be
a more efficient way to chain a number of small image operations, as
for example Ingest\_AT would do if region selection and primary beam
correction are used. 




\clearpage
