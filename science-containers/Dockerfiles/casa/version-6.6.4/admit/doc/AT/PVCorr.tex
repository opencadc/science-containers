% in RST now, don't edit here

\subsection{PVCorr\_AT}

\subsubsection{Description}

PVCorr\_AT uses a position-velocity slice to correlate distinct repeating
structures in this slice along the velocity (frequency) direction to find
emission or absorbtion lines.

Caveat: if the object of interest has very different types of emission regions,
e.g a nuclear and disk component which do not overlap, the detection may not work
as well.

\subsubsection{Use Case}

For weak lines (NGC253 has nice examples of this) none of the CubeStats or CubeSpectrum give
a reliable way to detect such lines, because they are essentially 1-dimensional cuts through the
cube and become essentially like noise.  
When looking at a Position-Velocity slice, weak lines show up quite clearly to the eye,
because they are coherent 2D structures which mimick those of more obvious and stronger lines
in the same PVSlice. The idea is to cross-correlated an area around such strong lines along the
frequency axis and compute a cross-correlation coefficient.

There is also an interesting use case for virtual projects: imagine a PV slice with only
a weak line, but in another related ADMIT object (i.e. a PVslice from another spectrum window) it 
is clearly detected. Formally a cross correlation can only be done in velocity space when the
VLSR is known, but within a spectral window the non-linear effects are small. Borrowing a
template from another spw with widely different frequencies should be used with caution.

\subsubsection{Input BDPs:}

\begin{description}

\item[PVSlice] The input PVSlice

\item[PVslice-2] Alternative PVSlice (presumably from another virtual project)

\item[CubeStats] Statistics on the parent cube

\end{description}


\subsubsection{Input keywords}

\begin{description}

\item[cutoff] a conservative cutoff above which an area is defined for the N-th strongest line.
Can also be given in terms of sigma in the parent cube.

\item[order]  Pick the N-th strongest line in this PV slice as the template.
Default:1 

\end{description}

\subsubsection{Output BDPs}

\begin{description}

\item[PVCorr\_BDP] Table with cross-correlation coefficients

\end{description}


\subsubsection{Procedure}

After a template line is identified  (usually the strongest line in the PV Slice), a conservative 
polygon (not too low a cutoff) 
is defined around this, and cross correlated along the frequency axis. Currently this 
needs to be a single polygon (really?)

Given the odd shape of the emission in a PVSlice, the correlation coefficient is not
exactly at the correct ``velocity''.  A correction factor needs to be determined
based on identifying the template line with a known line frequency, and VLSR. Since
line identifaction is the next step after this, this catch-22 situation needs to
be resolved, otherwise a small systemic offset can be present.



\subsubsection{CASA tasks used}

none exist yet that can do this. NEMO has a program written in C, and the idea in there will
need to be ported to python.


\clearpage
