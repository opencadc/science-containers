% in RST now, don't edit here

\subsection{CubeStats\_AT}

\subsubsection{Description}

CubeStats\_AT will compute per-channel statistics, and can help
visualize what spectral lines there are in a cube. 
They are stored in a CubeStats\_BDP.
For certain types of cubes, these can be effectively used in LineID\_AT to identify
spectral lines. In addition, tasks such as CubeSum\_AT and LineCube\_AT can use this if the noise
depends on the channel.


\subsubsection{Use Case}

Many programs needs to know (channel dependent) statistics from a cube in order to be able to
clip out the noise and get as much signal in the processing pipeline. In addition, the analysis
to compute channel based statistics needs to include some robustness, to remove signal. This
will become progressively hard if there is a lot of signal in the cube.

\subsubsection{Input BDPs}

\begin{description}

\item[SpwCube\_BDP] input spectral window cube, for example as created with Ingest\_AT.

\end{description}


\subsubsection{Input Keywords}

\begin{description}

\item[robustness] Several options can be given here, to select the robust RMS.
(negative half gauss fit, robust, ...)

\end{description}

\subsubsection{Output BDPs}

\begin{description}

\item[CubeStats\_BDP] A table containing various quantities (min,max,median,rms) on a per channel basis.
A global statistics can (optionally?) also be recorded. The graphical output contains at least two plots:
1) histogram of the distribution of Peak (P), Noise (N), and P/N, typically logarithmically; 
2) A line diagram shows the P, N and P/N (again logarithmically) as function of frequency.
As a good diagnostic, the P/N should now not depend on frequency if only line-free channels
are compared. It can also be a good input BDP to LineID\_AT.

\end{description}



\subsubsection{Procedure}

The imstat task (or the ia.statistics tool) in CASA can compute plane based statistics. Although
the  {\it medabsdevmed} output key is much better than just looking at {\it sigma}, 
a signal clipping robust option may soon be available to select a more robust way to compute what
we call the Noise column in our BDP. Line detection (if this BDP is used) is all based
on log(P/N).

\subsubsection{CASA tasks used}

\begin{description}

\item[imstat] extract the statistics per channel, and per cube. Note that
the tool and task have different capabilities, for us, we need to the tool,
but this does not have the new robust option yet.

\end{description}

\clearpage
