% in RST now, don't edit here

\subsection{Flow11\_AT, Flow1N\_AT, FlowN1\_AT}


\subsubsection{Description}


The ``Flow'' series, together with File\_AT, perform nothing more than
converting a dummy (one or more) input BDP(s) into (one or more) output BDP(s).
Optionally the file(s) associated with these BDP's can be created as zero
length files.


\subsubsection{Use Case}

Useful to benchmark the basic ADMIT infrastucture cost of a (complex)
flow.  A simple Flow11\_AT can be used for converting a single
File\_BDP. Two variadic versions are available:
Flow1N\_AT with 1 input and N outputs, and FlowN1\_AT with N inputs
and 1 output.


\subsubsection{Input BDPs}

\begin{description}

\item[File\_BDP] containing simply the string pointing to a file. This
  file does not have to exist. Flow1N\_AT can handle multiple input
  BDPs.

\end{description}

\subsubsection{Input Keywords:}

\begin{description}

\item[file] The filename of the output object if there is only one
  output BDP (Flow11\_AT and FlowN1\_AT). For Flow1N\_AT the filename
  is generated by adding an index 1..N to the input filename, so this
  keyword will be absent in this AT.

\item[n] Normally variadic input or output can be determined otherwise
  (e.g. it generally depends in a complex way on user parameters), but
  for the Flow series it has to be manually set.  Only allowed for
  Flow1N\_AT where the ``file'' keyword is not use.  Default: 2.

\item[touch]  Update timestamp (or create zero length file
  if filename did not exist)? Default: False.

\end{description}

\subsubsection{Output BDPs:}

\begin{description}
\item[File\_BDP] containing simply the string pointing to the
  file, following the same convention the other file containers
(Image\_BDP, Table\_BDP) do, except their overhead.
There is no formal requirement this should be a relative or absolute
address. A symlink is allowed where the OS allows this, a zero length
file is also allowed.
\end{description}

\subsubsection{Procedure}

Only Unix tools, such as {\bf touch}, are used.
There are no CASA dependencies, and thus no CASA tasks used.

% \subsubsection{CASA tasks used}
% none


\clearpage
