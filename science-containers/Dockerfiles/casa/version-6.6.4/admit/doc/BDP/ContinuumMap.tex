\subsection{ContinuumMap}
\begin{description}
\item[Type:] ContinuumMap
\item[Description:] 
This is an image (2D or 3D) of the continuum level of the field in question.
It is computed where there is no spectral line emission present.  

For the subsequent case of continuum subtraction from a spectral line cube,
we should compute some confidence level there is continuum in this data.
Note the RMS can depend on the channel across the band, and this should
be taken into account as well.

\item[Constituents:] BDP\_Image

\item[ADMIT Task:] AT\_ContinuumMap

\item[CASA Task(s):] TBD

\item[Input BDP(s):] Spectral Cube. Optionally, LineList, if it is known that
the cube contains line emission, to exclude such.

\item[Output BDP(s):]  
  2D or 3D image of continuum.  
  An summary indication of the continuum level. Should we automatically
  run CubeStats on the image?
\end{description}
