\subsection{CubeSpectrum}
\begin{description}
\item[Type:] CubeSpectrum
\item[Description:]

CubeSpectrum will contain a table of a spectrum through the whole cube, at a
single point (with optional averaging over a region or channels)?.

This differs from the CubeStats where in theory one can also plot "a
spectrum" where the brightest value "anywhere" in each channel is plotted
vs. frequency.

This was previously called PeakSpectrum.  This version has a keyword
describing where this point was taken.

NOTE: In principal, CubeSpectrum is redundant with SpectralMap. A CubeSpectrum
is just a SpectralMap where N=1.  So we should consider deprecating CubeSpectrum.

\item[Constituents:] BDP\_Table

\item[ADMIT Task:] AT\_CubeSpectrum \\
  Keywords: \\
   - location where spectrum is to be taken \\
   - [optional region/channels to average?]

\item[CASA Task(s):]  TBD

\item[Input BDP(s):] A spectral cube 

\item[Output BDP(s):]

A table composed of two columns: frequency (in GHz, middle of channel)
   and intensity (Jy/Beam)
\end{description}
