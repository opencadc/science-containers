\subsection{CubeSpectrum}
\begin{description}
\item[Type:] CubeSpectrum
\item[Description:]

Produces per channel statistics on an input image cube. Output
format is a table with data described below.  In addition to
per channel output, (optional?) per cube values can also be
computed.

\item[ADMIT Task:] AT\_CubeStats \\

  Keywords: TBD

\item[Constituents:] BDP\_Table

\item[CASA Task(s):]  imstat

\item[Input BDP(s):]
Spectral cube (1 or more planes) 

\item[Output BDP(s):]
\begin{verbatim}
- a table with channel dependent values:
    channel number
    freq in middle of channel
    mean 
    sigma (not rms, sigma is around the mean)
    max value (i.e. peak)
    maxposx (optional)
    maxposy (optional)

- a global value for the cube
    mean
    sigma
    max
    maxposx 
    maxposy 
    maxposz

   This global value may be needed for CubeSpectrum

Note the difference between sigma and rms.  rms contains a bias, sigma
is N-1 normalized, rms is N normalized

######################################################################
OTHER NOTES:

# confirmed help(imstat) and this list are the same.
# ia.statistics() will by default give 17 (robust=F is default)
# is will not do median, medabsdevmed, quartile
# imstat produces a dictionary of 20 items
# (used to be 16; the minpos[f],maxpos[f] were added recently)
# !! Note that currently minpos[f],maxpos[f] are only available per cube,
#    not per slice

  CASA                      MIRIAD         NEMO
rms
medabsdevmed  "MADM"
minpos        (new)
min                         Minimum        min
max                         Maximum        max
sum                         Sum            sum
minposf       (new)
median                                     med
flux
sumsq
maxposf
trcf
quartile
npts                        Npoints        N
maxpos
mean                        Mean           mean
sigma                       rms            sigma
trc
blc
blcf


channel       -             plane          iz
velocity      -             Velocity       z
frequency     -             -              z
skewness      -             -              skew
kurtosis      -             -              kurt


imstat() and ia.statistics() will have new options to compute
statistics based on "robustly" removing signal, or finding
the normal part of the noise distribution, by using a new 
filter= keyword.
Three methods ("hinges-fences", "fit-half", "chauvenet") will
initially be implemented.


Q: do we need to check the header for a beam, and do a proper
points-per-beam correction?
\end{verbatim}
\end{description}
